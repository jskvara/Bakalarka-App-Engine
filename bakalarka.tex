%% History:
% Pavel Tvrdik (26.12.2004)
%  + initial version for PhD Report
%
% Daniel Sykora (27.01.2005)
%
% Michal Valenta (3.12.2008)
% rada zmen ve formatovani (diky M. Duškovi, J. Holubovi a J. Žďárkovi)
% sjednoceni zdrojoveho kodu pro anglickou, ceskou, bakalarskou a diplomovou praci

% One-page layout: (proof-)reading on display
%%%% \documentclass[11pt,oneside,a4paper]{book}
% Two-page layout: final printing
\documentclass[11pt,twoside,a4paper]{book}   
%=-=-=-=-=-=-=-=-=-=-=-=--=%
% The user of this template may find useful to have an alternative to these 
% officially suggested packages:
\usepackage[czech, english]{babel}
\usepackage[T1]{fontenc} % pouzije EC fonty 
% pripadne pisete-li cesky, pak lze zkusit take:
% \usepackage[OT1]{fontenc} 
\usepackage[utf8]{inputenc}
%=-=-=-=-=-=-=-=-=-=-=-=--=%
% In case of problems with PDF fonts, one may try to uncomment this line:
%\usepackage{lmodern}
%=-=-=-=-=-=-=-=-=-=-=-=--=%
%=-=-=-=-=-=-=-=-=-=-=-=--=%
% Depending on your particular TeX distribution and version of conversion tools 
% (dvips/dvipdf/ps2pdf), some (advanced | desperate) users may prefer to use 
% different settings.
% Please uncomment the following style and use your CSLaTeX (cslatex/pdfcslatex) 
% to process your work. Note however, this file is in UTF-8 and a conversion to 
% your native encoding may be required. Some settings below depend on babel 
% macros and should also be modified. See \selectlanguage \iflanguage.
%\usepackage{czech}  %%%%%\usepackage[T1]{czech} %%%%[IL2] [T1] [OT1]
%=-=-=-=-=-=-=-=-=-=-=-=--=%

%%%%%%%%%%%%%%%%%%%%%%%%%%%%%%%%%%%%%%%
% Styles required in your work follow %
%%%%%%%%%%%%%%%%%%%%%%%%%%%%%%%%%%%%%%%
\usepackage{graphicx}
%\usepackage{indentfirst} %1. odstavec jako v cestine.

\usepackage{k336_thesis_macros} % specialni makra pro formatovani DP a BP
 % muzete si vytvorit i sva vlastni v souboru k336_thesis_macros.sty
 % najdete  radu jednoduchych definic, ktere zde ani nejsou pouzity
 % napriklad: 
 % \newcommand{\bfig}{\begin{figure}\begin{center}}
 % \newcommand{\efig}{\end{center}\end{figure}}
 % umoznuje pouzit prikaz \bfig namisto \begin{figure}\begin{center} atd.

% Zdrojaky
\usepackage{listings}
\lstset{language=Java}
\lstset{basicstyle=\footnotesize\ttfamily}
\lstset{numbers=left}
\lstset{breaklines=true}
\lstset{frame=single}
\lstset{tabsize=4}



%%%%%%%%%%%%%%%%%%%%%%%%%%%%%%%%%%%%%
% Zvolte jednu z moznosti 
% Choose one of the following options
%%%%%%%%%%%%%%%%%%%%%%%%%%%%%%%%%%%%%
% \newcommand\TypeOfWork{Diplomová práce} \typeout{Diplomova prace}
% \newcommand\TypeOfWork{Master's Thesis}   \typeout{Master's Thesis} 
\newcommand\TypeOfWork{Bakalářská práce}  \typeout{Bakalarska prace}
% \newcommand\TypeOfWork{Bachelor's Project}  \typeout{Bachelor's Project}


%%%%%%%%%%%%%%%%%%%%%%%%%%%%%%%%%%%%%
% Zvolte jednu z moznosti 
% Choose one of the following options
%%%%%%%%%%%%%%%%%%%%%%%%%%%%%%%%%%%%%
% nabidky jsou z: http://www.fel.cvut.cz/cz/education/bk/prehled.html

%\newcommand\StudProgram{Elektrotechnika a informatika, dobíhající, Bakalářský}
%\newcommand\StudProgram{Elektrotechnika a informatika, dobíhající, Magisterský}
% \newcommand\StudProgram{Elektrotechnika a informatika, strukturovaný, Bakalářský}
 %\newcommand\StudProgram{Elektrotechnika a informatika, strukturovaný, Navazující magisterský}
\newcommand\StudProgram{Softwarové technologie a management, Bakalářský}
% English study:
% \newcommand\StudProgram{Electrical Engineering and Information Technology}  % bachelor programe
% \newcommand\StudProgram{Electrical Engineering and Information Technology}  %master program


%%%%%%%%%%%%%%%%%%%%%%%%%%%%%%%%%%%%%
% Zvolte jednu z moznosti 
% Choose one of the following options
%%%%%%%%%%%%%%%%%%%%%%%%%%%%%%%%%%%%%
% nabidky jsou z: http://www.fel.cvut.cz/cz/education/bk/prehled.html

%\newcommand\StudBranch{Výpočetní technika}   % pro program EaI bak. (dobihajici i strukt.)
%\newcommand\StudBranch{Výpočetní technika}   % pro prgoram EaI mag. (dobihajici i strukt.)
\newcommand\StudBranch{Softwarové inženýrství}            %pro STM
%\newcommand\StudBranch{Web a multimedia}                  % pro STM
%\newcommand\StudBranch{Computer Engineering}              % bachelor programe
%\newcommand\StudBranch{Computer Science and Engineering}  % master programe


%%%%%%%%%%%%%%%%%%%%%%%%%%%%%%%%%%%%%%%%%%%%
% Vyplnte nazev prace, autora a vedouciho
% Set up Work Title, Author and Supervisor
%%%%%%%%%%%%%%%%%%%%%%%%%%%%%%%%%%%%%%%%%%%%

\newcommand\WorkTitle{Ověření škálovatelnosti Google App Engine aplikací	}
\newcommand\FirstandFamilyName{Jakub Škvára}
\newcommand\Supervisor{Ing. Marek Šmíd}


% Pouzijete-li pdflatex, tak je prijemne, kdyz bude mit vase prace
% funkcni odkazy i v pdf formatu
\usepackage[
pdftitle={\WorkTitle},
pdfauthor={\FirstandFamilyName},
bookmarks=true,
colorlinks=true,
breaklinks=true,
urlcolor=red,
citecolor=blue,
linkcolor=blue,
unicode=true,
]
{hyperref}



% Extension posted by Petr Dlouhy in order for better sources reference (\cite{} command) especially in Czech.
% April 2010
% See comment over \thebibliography command for details.

\usepackage[square, numbers]{natbib}             % sazba pouzite literatury
%\usepackage{url}
%\DeclareUrlCommand\url{\def\UrlLeft{<}\def\UrlRight{>}\urlstyle{tt}}  %rm/sf/tt
%\renewcommand{\emph}[1]{\textsl{#1}}    % melo by byt kurziva nebo sklonene,
\let\oldUrl\url
\renewcommand\url[1]{<\texttt{\oldUrl{#1}}>}




\begin{document}

%%%%%%%%%%%%%%%%%%%%%%%%%%%%%%%%%%%%%
% Zvolte jednu z moznosti 
% Choose one of the following options
%%%%%%%%%%%%%%%%%%%%%%%%%%%%%%%%%%%%%
\selectlanguage{czech}
%\selectlanguage{english} 

% prikaz \typeout vypise vyse uvedena nastaveni v prikazovem okne
% pro pohodlne ladeni prace


\iflanguage{czech}{
	 \typeout{************************************************}
	 \typeout{Zvoleny jazyk: cestina}
	 \typeout{Typ prace: \TypeOfWork}
	 \typeout{Studijni program: \StudProgram}
	 \typeout{Obor: \StudBranch}
	 \typeout{Jmeno: \FirstandFamilyName}
	 \typeout{Nazev prace: \WorkTitle}
	 \typeout{Vedouci prace: \Supervisor}
	 \typeout{***************************************************}
	 \newcommand\Department{Katedra kybernetiky}
	 \newcommand\Faculty{Fakulta elektrotechnická}
	 \newcommand\University{České vysoké učení technické v Praze}
	 \newcommand\labelSupervisor{Vedoucí práce}
	 \newcommand\labelStudProgram{Studijní program}
	 \newcommand\labelStudBranch{Obor}
}{
	 \typeout{************************************************}
	 \typeout{Language: english}
	 \typeout{Type of Work: \TypeOfWork}
	 \typeout{Study Program: \StudProgram}
	 \typeout{Study Branch: \StudBranch}
	 \typeout{Author: \FirstandFamilyName}
	 \typeout{Title: \WorkTitle}
	 \typeout{Supervisor: \Supervisor}
	 \typeout{***************************************************}
	 \newcommand\Department{Department of Computer Science and Engineering}
	 \newcommand\Faculty{Faculty of Electrical Engineering}
	 \newcommand\University{Czech Technical University in Prague}
	 \newcommand\labelSupervisor{Supervisor}
	 \newcommand\labelStudProgram{Study Programme} 
	 \newcommand\labelStudBranch{Field of Study}
}

%%%%%%%%%%%%%%%%%%%%%%%%%%    Titulni stranka / Title page 

\coverpagestarts

%%%%%%%%%%%%%%%%%%%%%%%%%%%    Podekovani / Acknowledgements 

\acknowledgements
\noindent
Chtěl bych poděkovat svému vedoucímu panu Ing. Marku Šmídovi, za ochotu a pomoc při psaní této bakalářské práce.

%%%%%%%%%%%%%%%%%%%%%%%%%%%   Prohlaseni / Declaration 

\declaration{V~Praze dne 27.\,5.\,2011}
%\declaration{In Kořenovice nad Bečvárkou on May 15, 2008}


%%%%%%%%%%%%%%%%%%%%%%%%%%%%    Abstract 
 
\abstractpage

This bachelor thesis is describing advantages and disadvantages of cloud computing. Which appliactaions are useseful to place to cloud and also limitations which cloud brings. In part one, I am describing what is cloud generally and introduce various possibilities and compare them to readers. I have choosen App Engine from Google for my research, therefore I am describing this cloud in more detail.
 
In part two I am describing programming a several small appliacations for testing and comparing performance of storage. Then I made one larger appliacation and described its structure and architecture. I tested scalability and performance of this appliacation with large amount of requests. All results are summarized and compared in conclusion.

% Prace v cestine musi krome abstraktu v anglictine obsahovat i
% abstrakt v cestine.
\vglue60mm

\noindent{\Huge \textbf{Abstrakt}}
\vskip 2.75\baselineskip

\noindent
Tato bakalářská práce popisuje výhody a nevýhody použití cloudu. Které aplikace se hodí pro umístění do cloudu a také omezení které cloud přináší. V první části popisuji co je to cloud obecně a seznamuji čtenáře s různými možnostmi a porovnávám je. Pro ověření jsem si vybral App Engine od firmy Google a popisuji tento cloud podrobněji.

V durhé části popisuji jak jsem naprogramoval několik menších aplikací pro otestování a porovnání rychlosti uložiště. Dále jsem napsal jednu větší aplikaci a popsal její strukturu a architekturu. Na této aplikaci jsem pomocí velkého počtu požadavků otestoval škálovatelnost a 
rychlost App Engine. Všechny výsledky jsem přehledně shrnul a porovnal v závěru. 


%%%%%%%%%%%%%%%%%%%%%%%%%%%%%%%%  Obsah / Table of Contents 

\tableofcontents


%%%%%%%%%%%%%%%%%%%%%%%%%%%%%%%  Seznam obrazku / List of Figures 

\listoffigures


%%%%%%%%%%%%%%%%%%%%%%%%%%%%%%%  Seznam tabulek / List of Tables

\listoftables


%**************************************************************

\mainbodystarts
% horizontalní mezera mezi dvema odstavci
%\parskip=5pt
%11.12.2008 parskip + tolerance
\normalfont
\parskip=0.2\baselineskip plus 0.2\baselineskip minus 0.1\baselineskip

% Odsazeni prvniho radku odstavce resi class book (neaplikuje se na prvni 
% odstavce kapitol, sekci, podsekci atd.) Viz usepackage{indentfirst}.
% Chcete-li selektivne zamezit odsazeni 1. radku nektereho odstavce,
% pouzijte prikaz \noindent.

%*****************************************************************************
\cleardoublepage
\chapter{Úvod: Výběr tématu}

\section{Co je to cloud}
Pro mou bakalářskou práci jsem si vybral poměrně nové a nezmapované téma a to popis cloudových služeb. Jedná se o nový způsob hostování internetových aplikací a ukládání dat na webu. U klasického způsobu ukládání dat se použije jeden nebo více nezávislých serverů a na ně se nahraje ve většině případů pouze jedna aplikace. Cloud nám přináší nový přístup, místo abychom měli jeden stroj, použijeme více strojů dohromady, které budou spolupracovat. Aplikace žádný rozdíl nepozná a může zde zároveň běžet mnoho programů. 

Důležitým důvodem k využívání cloudů je větší efektivita využití hardware. Pokud je aplikace málo používaná, například v nočních hodinách, jsou zdroje využívány jinými aplikacemi, které mohou obsluhovat uživatele z jiné části světa, kde je třeba odpoledne. Nebo pokud je aplikace vytížena a nestíhá, může systém spustit další instanci té samé aplikace. O rozložení zátěže a správu počtu aplikací se stará cloud samotný.

Další výhodou je možnost dynamicky navyšovat hardwarové parametry infrastruktury přidáváním strojů bez nutnosti přerušit provoz. Cena hardwaru se postupem času snižuje, podle Mooreova zákona\footnote{Mooreův zákon --- http://en.wikipedia.org/wiki/Moore's\_law} z roku 1965 se složitost součástek každé dva roky zdvojnásobí při zachování stejné ceny. Tento zákon platí dodnes a předpokládá se, že bude platit minimálně do roku 2015 až 2020. S možností dynamicky přidávat hardware budeme připraveni rozšiřovat infrastrukturu podle potřeby. O toto se ale starají společnosti poskytující cloudy, nám tedy odpadá nutnost starat se o hardware. 

Nevýhodou cloudu je, že nemáme hardware plně pod kontrolou a musíme se spolehnout na společnost poskytující hosting. Pokud bychom chtěli provozovat vlastní cloud, museli bychom mít složitou infrastrukturu a vyvinout vlastní řešení pro efektivní rozložení zátěže všech aplikací. Takovouto možnost má jen několik společností na světě, které patří k těm největším.

\section{Motivace pro toto téma}
Hlavní motivací k výběru tohoto tématu bylo, že se v poslední době stávájí cloudy čím dál tím více používanější. Jedná se o úplně nový přístup a tak o cloudech zatím nenajdeme mnoho zdrojů. Přišlo mi zajímavé vyzkoušet a prověřit možnosti jednoho z těchto cloudů. Jak se bude chovat při vysokém počtu požadavků, kde jsou limity takového cloudu a v neposlední řadě kolik hardwarových prostředků bude při zátěži spotřebováno a jaká bude cena této služby.

\section{Průběh testování}
V mé bakalářké práci jsem ověřil a porovnal rychlost různých řešení práce s uložištěm dat na cloudu. Pro tento test jsem připravil jednoduchou aplikaci. Poté jsem napsal větší a složitější aplikaci, kterou jsem následně otestoval vysokým počtem požadavků. Porovnával jsem jak se bude cloud chovat a jak zařídí rozložení zátěže.

%*****************************************************************************
\cleardoublepage
\chapter{Teorie: Obecný popis cloudu}

\section{Novinka jménem cloud computing}
V poslední době se čím dál tím více začíná mluvit o cloud computingu. Jedná se o nový typ hostingu a ukládání webového obsahu vůbec. Oproti klasickému způsobu, kde máme jeden konkrétní server, na určeném místě, se svojí danou pamětí, procesorem a pevným diskem, nám tento nový přístup umožňuje nezabývat se hardwarem, ale mluvíme takto o platformě. 

Definice se dosti různí, takže použiji verzi od Národního institutu standardů a technologií (National Institute of Standards and Technology) \cite{nist-cloud},
%http://csrc.nist.gov/groups/SNS/cloud-computing/cloud-def-v15.doc
%http://google-cz.blogspot.com/2011/04/clanek-bezpecnost-citlivych-dat-v.html
která volně přeložena zní: cloud computing je způsob poskytování sdílených škálovatelných zdrojů (výpočetní kapacita, uložiště, služby, aplikace, ...), k nímž je přistupováno skrz síť a které jsou uživateli k dispozici ihned na vyžádání. 

Mezi hlavní výhody je považováno snižování nákladů a zvyšování efektivity. Nemusíme vlastnit hardware, za jehož pořízení a správu je potřeba vynaložit nemalé finanční náklady, přičemž většina zdrojů není plně zatížena. Zvyšování efektivity se projevuje hlavně placením jen za využité zdroje. Pokud bychom měli vlastní infrastrukturu, tak v době nižší aktivity nevyužíváme možnosti serverů naplno a platíme vlastně za nevyužité zdroje. Naopak v cloudu jsou naše prostředky sdíleny s ostatními a v době neaktivity můžou být nabídnuty někomu jinému.

\section{Infrastructure as a Service, Platform as a Service, Software as a Service}
Existují různé nabídky cloudových řešení pro efektivní využívání zdrojů hardware pro více apliací. Nejzákladnější je Iaas - Infrastructure as a Service (Infrastruktura jako Služba) - jedná se například o Amazon EC2 \cite{amazon-ec2} cloud, kde platíme jen za spotřebované zdroje, které reálně využijeme a na hardware si můžeme sami instalovat co potřebujeme. 

U PaaS - Platform as a Service  (Platforma jako Služba) již nemáme přístup k hardwaru, to znamená že nelze instalovat žádný software, ale máme zde připravená API pro různé služby které můžeme využívat a většinou i další nástroje pro vývoj na lokálním stroji a pro deployment.

Nejvíce jsme od fungování služby odstíněni u SaaS - Software as a Service (Software jako Služba) - jedná se například o online e-mailové služby jako \verb|gmail.com| anebo \verb|seznam.cz|, obrázkové galerie jako \verb|flickr.com| anebo \verb|rajce.cz|, tedy služby které používáme prostřednictvím internetu a nezajímá nás jak a kde jsou data uložena a nemáme ponětí jak jsou naprogramované.

\section{Horizontální a vertikální škálování}
Nové služby a především sociální sítě s rychlým nárůstem uživatelů a potřebě dynamicky měnit počet serverů donutily programátory a správce přemýšlet o novém způsobu ukládání a organizace dat. Pokud náš server nestíhá, tak máme dvě možnosti, jak tento problém řešit. Prvním z řešení je vertikální škálování, to znamená že koupíme silnější hardware, přidáme procesor, paměť a další komponenty podle potřeby. Nevýhodou tohoto řešení ale je, že takto nejde infrastruktura rozšiřovat do nekonečna, protože po čase narazíme na hardwarové limity. 

Druhou z možností je nakoupení více serverů, nemusí být ani velmi výkonné, ale řešení spočívá v propojení těchto počítačů dohromady, címž můžeme zvětšovat naši infrastrukturu bez omezení. Můžeme toto řešení přirovnat k problému z běžného života, kdy potřebujeme převézt určitý počet osob z jednoho místa na druhé. Můžeme objednat autobus, který jednoduše řeší tento problém. Pokud počet osob naroste, tak můžeme objednat autobus s větší kapacitou, jenže pokud se bude počet osob zvyšovat, tak časem již nenajdeme tak velký autobus pro přepravu všech osob. Takže jako řešení budeme muset objednat více vozidel, ale ty poté budeme moci efektivně zaplňovat podle počtu osob. Nevýhoda tohoto řešení spočívá v složitějším správě infrastruktury, potřebujeme software, který se stará o propojení, synchronizaci a spolupráci všech částí systému, nebo pokud nějáký stroj přestane fungovat, musíme zajistit aby ho ihned zastoupil jiný se stejnou funkcí jako předchozí.

Pro cloud computing se obecně vžila značka mraku (v angličtině znamená cloud mrak) a vznikla proto, že na obrázcích a schématech se ve většině případů mrakem značí internet a vzdálená zařízení, které nejsou uloženy u nás. A to právě z toho důvodu, že jsou tyto služby většinou přístupné skrze internet a přistupujeme k nim vzdáleně.

\section{Různé druhy pohledu na cloud computing}
Cloud computing se jako každá jiná novinka potýkal s různými názory od těch pozitivních až po ty z negativní. Někteří tvrdili že se jedná jen o buzzword\footnote{módní slovo}, který má nalákat nové zákazníky, jiní predikovali že se takovýto princip nemůže nikdy uchytit. Zde je pár výroků  z doby, kdy nebyl tolik rozšířený:

\begin{quotation}
The interesting thing about Cloud Computing is that we’ve redefined Cloud Computing to include everything that we already do...  I don’t understand what we would do differently in the light of Cloud Computing other than change the wording of some of our ads.

\em Larry Ellison, Oracle Corporation CEO, Wall Street Journal, 26. září 2008
\end{quotation}

Larry Ellison říká, že cloud computing je jen pojmenování toho, co již dávno používáme a že jediná změna, která je potřeba je změna textů u reklam. Je pravda, že některé velké společnosti, jako třeba Oracle anebo Google používali tento přístup již mnohem dříve než vzniknul samotný název, ale pravdou je, že v posledních dvou letech se začal cloud computing používat masivněji a to hlavně díky možnosti pronájmu cloudů. Nyní si mohou programátoři vyzkoušet pracovat s cloudy a použít je i pro své menší aplikace, bez nutnosti spravovat a starat se o velké množství strojů.

\begin{quotation}
A lot of people are jumping on the (cloud) bandwagon, but I have not heard two people say the same thing about it. There are multiple definitions out there of “the cloud.”

\em Andy Isherwood, HP’s vice president, ZDnet News, 11. prosinec 2008
\end{quotation}

Andy Isherwood naznačuje, že není přesně daná definice toho, co ještě cloud je a co už není. Je to způsobeno tím, že po vzniku tohoto názvu chtěl každý s více než jedním serverm označovat svoje služby jako cloudové. To vedlo spíše ke zmatení, ale v poslední době se toto slovní spojení ustálilo pro farmu serverů se snadnou škálovatelností a jednoduchou možností přidat novou instanci.

\begin{quotation}
It’s stupidity. It’s worse than stupidity: it’s a marketing hype campaign. Somebody is saying this is inevitable — and whenever you hear somebody saying that, it’s very likely to be a set of businesses campaigning to make it true.

\em Richard Stallman, Founder of GNU Project and Free Software Foundation, The Guardian, 29. září 2008
\end{quotation}

Richard Stallman má na cloud poněkud negativní názor a pro The Guardian vyslovil názor že se jedná o hloupost a jde jen o nafouknutou bublinu podpořenou businessovými kampaněmi. Kritizoval hlavně uložení dat mimo naši vlastní kontrolu a nutnost spolehnutí se na společnost, které dávame naše data a aplikace k dispozici. Nikdo nám nemůže na sto procent zaručit, že bude tato společnost fungovat i po několika letech. Navíc jsme většinou vázáni na API rozhraní, služby a možnosti platformy určené danou společností.

Pokud budeme chtít přenést službu k jinému poskytovateli cloudu, bude nám to s největší pravděpodobností činit nemalé potíže, v angličtině se používá termín \emph{lock-in}, což znamená doslova zamknutí. Sice v poslední době vznikají návrhy na jednotná rozhraní a sjednocení rozhraní těchto služeb, aby byl přechod co nejjednodušší, ale ty se bohužel zatím nesetkaly s větším rozšířením. Do budoucna by to mohla být jedna z klíčových vlastností při rozhodování kterou službu zvolit.

\section{Porovnání Amazon Web Services, Windows Azure a Google App Engine}
Největší konkurenti Google App Engine jsou Amazon Web Services - EC2 (Elastic Compute Cloud) a Microsoft Azure. Pokud budemem mít zakázku, pro kterou je nejvýhodnější použit cloudovou infrastrukturu, budeme se pravděpodobně rozhodovat mezi těmito třemi, jedná se o velké a známé společnosti s rozsáhlým zázemím. Je tedy málo pravděpodobné, že by ze dne na den přestaly provozovat svoje služby, což může být velký problém u menších nebo méně znamých společnotí.

\subsection{Amazon Web Services - Elastic Compute Cloud}
Amazon EC2 je spíše blíže modelu IaaS, takže dostaneme hardware s kterým si můžeme dělat co chceme, instalovat libovolný software a musíme si ho sami spravovat. Největší výhodou je rychlé přidávání nových serverů kdykoliv potřebujueme a platba jen za spotřebované zdroje. Navíc není tento cloud vázán žádnými API a omezeními, takže pokud budeme chtít přesunout naši aplikaci na náš server, tak nebudeme muset měnit aplikaci a to platí i naopak, tedy pro přesun aplikace na cloud. Jedná se čistě o pronájem hardwaru. Neýhoda Amazon Web Services je v tom, že platíme hned jakmile nahrejeme naši aplikaci, neexistují žádné volné kvóty jako u App Engine. Amazon v rámci svých služeb nabízí i další možnosti, například speciální relační i nerelační uložiště a další produkty, celý seznam je možné najít na stránce \verb|http://aws.amazon.com/products/|. 

\subsection{Microsoft Azure}
Microsoft Azure je již více podobný App Engine, jedná se o PaaS, máme zde již připravené prostředí pro několik jazyků: .NET (C\# a VisualBasic), C++, PHP, Ruby, Python a Java. Výhodou je, že můžeme použít klasickou relační databázi nazvanou SQL Azure Database (SAD), což se vyplatí pokud migrujeme nějaký projekt postavený na relační databázi, ale tento typ je hůře škálovatelný. Vedle SAD můžeme použít i Azure Storage, která osahuje nerelační tabulky, tabulky pro velké objemy dat (blobs) a fronty (queues). Azure má speciální staging prostředí, kde můžeme přímo na cloudu vyvíjet aplikaci a nedostane se k ní nikdo, dokud není připravena na spuštení. V tomto prostředí také můžeme spouštět aplikaci v debug režimu, což nám umožňuje například nastavovat breakpointy, pokud potřebujeme aplikaci ladit přímo na cloudu. Azure také umožňuje propojení s Microsoft Live službami a s vývojovým prostředím Microsoft Visual Studio. Stejně jako u Amazonu platíme ihned jakmile nasadíme aplikaci do plného provozu, aby ji mohli vidět i ostatní. Azure je určitě výhodné použít, pokud vyvíjíme aplikace v technologiích od Microsoftu, naše infrastruktura je na těchto tehcnologiích založena anebo pokud používáme jako vývojový nástroj Visual Studio.

\subsection{Google App Engine}
Oproti dvěma předchozím má App Engine hlavní výhodu v tom, že nemusíme platit ihned jak aplikaci nahrajeme na cloud. Jsou zde nastavené kvóty, které jsou velmi vysoké a je potřeba velká návštěvnost pro jejich přesáhnutí, což je pro začínající aplikaci výhodné obzvláště v České republice. Takže pokud začínáme se startupem, nemusíme se v počátcích obávat velkých investic a pokud bude náš projekt úspěšný a bude obsluhovat velký počet požadavků, tak poté budeme muset platit jen za přesáhnuté limity, které se každých 24~hodin vynulují. Některé limity jsou nastaveny napevno a nejdou zvýšit ani za poplatek, je to kvůli tomu, aby se nemohlo stát, že jedna aplikace zaneprázdní celý cloud, což by mělo za následek zpomalení i ostatních aplikací. Tyto limity jsou naštěstí velmi vysoké, například datastore API má maximálně 141~241~791 volání za den a 784~676 volání za minutu. 
[tabulka kvót, cena]
Standardně se aplikace nahraje na adresu jako doména třetího řádu ve
tvaru \verb|jmeno-aplikace.appspot.com| a pokud potřebujeme můžeme nastavit i doménu vlastní. Další výhodou App Engine je možnost provozovat více verzí stejné aplikace najednou. Nahrají se pak jako poddomény, takže například \verb|verze.jmeno-aplikace.appspot.cz|. Tyto verze mohou běžet na cloudu zároveň a v administraci se dá nastavit, která bude výchozí. Toto do jisté míry nahrazuje staging area z Azure, výhodou zde je, že můžeme mít libovolný počet verzí. App Engine bohužel podporuje jen jazyky Java, Python a Go. Díky různým projektům, které jsou schopny přeložit další jazyky do javovského bytekódu, zde tedy můžeme spouštět velké množství dalších jazyků i když je to vykoupeno nižší rychlostí. Můžeme zde tedy používat jazyky jako: Groovy, Scala, Ruby s pomocí JRuby, PHP díky projektu Quercus (viz dále), JavaScript za pomoci Rhino a další. Jedním z důvodů přidání Javy do App Engine byla právě možnost běhu dalších jazyků nad Java Virtual Machine. Co se Javy týká, tak nejsou povoleny všechny třídy, nemůžeme například vytvářet nová vlákna, nemůžeme vytvářet nové soubory a některá volání třídy \verb|System| nedělají nic, například \verb|System.exit()| a \verb|Sytem.gc()|. Seznam všech povolených tříd je možné najít na adrese \verb|http://code.google.com/appengine/docs/java/jrewhitelist.html|. Kvůli těmto omezením bohužel nemůžeme použít všechny knihovny, anebo musíme použít upravenou verzi pro App Engine. Seznam nejpoužívanějších frameworků a knihoven s popisem zda jsou kompatibilní a případné nastavení pro App Engine je dostupný na adrese \verb|http://groups.google.com/group/google-appengine-java/web/will-it-play-in-app-engine?pli=1|. Na App Engine nemáme jistotu, že bude naše aplikace přímo připravena, protože kvůli tomu, aby se šetřily prostředky, jsou nahrány jen aktivní a využívané aplikace, se po určité době neaktivity aplikace odstaví a nahradí ji jiná. Můžeme si za poplatek zařídit, že naše aplikace bude vždy k dispozici, protože každé nahrání stojí čas. Můžeme tedy využívat většinu frameworků, ale kvůli tomuto nahrávání se každý kód navíc negativně projeví na době prvního přístupu k aplikaci, což je velmi znatelné především u rozsáhlých frameworků.

\section{API služby}
Abychom mohli propojit naši aplikaci s ostatními máme na App Engine velké množství API sloužících k různým účelům. Pojďme si nyní projít jaké možnosti máme. Některé z nich asi vůbec nevyužijeme, ale některé jsou velmi užitečné a důležité.

Asi jednou z nejpoužívanějších služeb, pokud pomineme Datastore, je Memcache. Jedná se o možnost, jak zrychlit častý přístup do databáze. Jedná se o key-value cache, která je přibližně desetkrát až stokrát rychejší, než přístup k Datastore. Nehodí se ale pro vše, protože data z ní po vypršení zmizí. Místo klasického cachování na disk, tedy máme možnost ukládat data do paměti. Implementace je podle standardu JSR-107, takže bude kompatibilní s dalšími knihovnami.

Mezi další užitečné služby patří Mail, posílání mailů funguje klasicky pomocí tříd javax.mail. Můžeme přidávat i přílohy. Některé soubory jsou z bezpečnostních důvodů zakázány, ale všechny běžně používané jsou povoleny. Přijímání e-mailů se ošetřuje pomocí servletu. Ve \verb|web.xml| se nastaví servlet pro url \verb|/_ah/mail/jmeno-emailu| a e-mail má tvar: \verb|jmeno-emailu@id-aplikace.appspot.com| a to bohužel i v případě, že máme nastavenou naši vlastní doménu. Příchozí e-mail se chová jako HTTP požadavek, takže v servletu se musíme zpracování postarat sami, podle toho co potřebujeme. 

Podobně jako Mail funguje i služba pro práci s XMPP protokolem\footnote{Extensible Messaging and Presence Protocol (http://xmpp.org/about-xmpp/technology-overview/)}. Jedná se o otevřený standardizovaný protokol Jabberu postavený na XML. Princip na App Engine je podobný jako s e-mailem, identifikátor příjemce je JID, který se dá získat  z e-mailu. Podporovány jsou i další funkce, jako posílání pozvánek, nastavování statusů a další. Přijímáme pomocí servletu nastaveného na adresu: \verb|/_ah/xmpp/message/chat/|.

Kvůli absenci JMS\footnote{Java Message Service (http://en.wikipedia.org/wiki/Java\_Message\_Service)} máme na App Engine možnost použít Task Queues. Jedná se o frontu úloh, které by mohly zpomalit náš systém, takže je výhodnější je zpracovat později. Fronta funguje následovně: pomocí Task Queues API přidáme do fronty URL naší aplikace, můžeme jí předávat parametry stejně jako u klasické URL\footnote{Uniform Resource Locator (http://en.wikipedia.org/wiki/Uniform\_Resource\_Locator)}. Provedení úlohy z fronty je záležitostí servletu\footnote{Servlet je komponenta napsaná v jazyce Java, určená pro spouštění na webovém serveru}, na který je URL nastavena pomocí \verb|web.xml|. Vykonání úlohy je omezeno deseti minutami, toto by měl být dostatečný limit pro běžné úlohy. Pokud servlet vrátí HTTP status mimo rozmezí 200 - 299, což znamená chybu, tak se úloha zavolá znovu aby proběhla v pořádku. Pokud potřebujeme informovat aplikaci o dokončení úlohy, musíme to řešit pomocí Datastore.

Pokud potřebujeme naše stránky integrovat s nějakou webovou službou, anebo používat veřejná REST API, použijeme URL Fetch. Jedná se o klasické java.net API, můžeme použít HTTP i HTTPS, většinu běžných portů a samozřejmě i všechny HTTP metody: GET, POST, PUT, HEAD i DELETE pro správné fungování REST rozhraní. K požadavku můžeme nastavovat i vlastní HTTP hlavičky.

Na některá data, jako například obrázky, nebo velké soubory se Datastore nehodí, maximální velikost entity je 1MB. Právě kvůli tomuto účelu můžeme na App Engine použít Blobstore, jedná se o uložiště pro velké soubory do velikosti až 2GB. Blobstore je plně oddělen od Datastore. Nahrávání je velice jednoduché, stačí použít forumář s prvkem \verb|<input type=”file” />| a atribut action formuláře nastavíme pomocí \verb|<%= blobstoreService.createUploadUrl("/upload")%>|.
Blobstore už se sám postará aby se soubor nahrál na správné místo. Zobrazovat soubory můžeme pomocí \verb|blobstoreService.serve(blob-key)|, potřebujeme k tomu klíč souboru. Tato služba umí vybrat všechny uložené soubory. Práce je velmi podobná jako s Datastore, akorát s tím rozdílem, že zde pracujeme s velkými soubory. Jedná se o užitečnou službu, protože dříve než existovala tato služba, se ukládání řešilo rozdělením do mnoha samostatných entit o velikosti 1MB a při zobrazení jsme je museli zase nazpět složit. Toto všechno se dělo na aplikační úrovni, takže jsme museli ošetřit všechny chybové případy a bylo vše velmi zdlouhavé a nepohodlné. Takto se o ukládání souborů stará AppEngine sám a nám stačí jednoduché API.

S předchozím Blobstore souvisí i další služba: Images. Jedná se o možnost úpravu obrázků přímo na serveru. Obrázky se načítají z Blobstore anebo můžeme službě předat přímo pole \verb|byte[]|. Můžeme takto aplikovat jednoduché transformace jako je změna velikosti, otočení, oříznutí, skládní obrázků a také magická funkce "I’m feeling lucky", která změní nastavení tmavých a světlých barev a k tomu také zvýší kontrast obrázku, výsledkem jsou pak více barevnější obrázky. Upravené obrázky můžeme přímo posílat uživetelům anebo uložit do Blobstore, pokud se budou zobrazovat čatěji. Služba obsahuje základní transformace, ale na vytvoření náhledů nebo menší úpravy jako zvětšení a zmenšení obrázku, které jsou pravěpodobně na webových stránkách nejpoužívanější, se Image API hodí výborně.

Pokud potřebujeme u naší aplikace vytvořit sekci jen pro přihlášené uživatele, nabízí nám k tomu App Engine možnost použít Users API a interní přihlašovací mechanismus Googlu využívaný u všech aplikací této společnosti, například tedy \verb|gmail.com|, \verb|youtube.com| a dalších. Použití je jednoduché, pokud uživatel není přihlášený, tak ho přesměrujeme na vygenerovanou přihlašovací stránku. Ta je stejná pro všechny služby Googlu, zadáme e-mail a heslo. Poté můžeme nastavit, které všechny údaje o sobě chceme aplikaci, do které se právě přihlašujeme, poskytnout. Nakonec nás stránka přesměruje zpět na naši aplikaci. Nyní můžeme o uživateli zjistit základní informace: přihlašovací e-mail a jednoznačný identifikátor ID. Odhlašování funguje stejně jako přihlašování, přesměrujeme uživatele na odhlašovací stránku Google, která nás následně po odhlášení znovu přesměruje, tentokrát na naši aplikaci.

Pokud chceme dát možnost přihlašování i pro uživatele, kteří nevlastní účet u Google, můžeme použít OAuth protokol. Ten není vázaný na konkrétní společnost, takže si můžeme vybrat kterého poskytovatele si vybereme. Jedná se o možnost přihlášení uživatelským jménem a heslem jiné aplikace a v naší aplikaci jen kontrolujeme token. Výhoda tohoto způsobu je, že uživateli stačí jeden účet pro více aplikací, nemusí si tak pamatovat hesla pro každou stránku na které má účet. S touto službou se pracuje velmi podobně jako s předchozím API.

App Engine obsahuje Capabilities API pomocí nějž můžeme zjistit, zda daná služba běží anebo ne. Můžeme tak ošetřit případ, kdy zrovna probíhá údržba anebo výpadek a služby jsou nedostupné. Jsou zde dostupné informace o těchto službách: Blobstore, čtení z Datastore, zápis do Datastore, Images, Mail, MemCache, TaskQueue, Url Fetch a XMPP.

Pro lepší spolupráci s klientskou stranou máme k dispozici Channel API. To se stará o trvalé spojení JavaScriptu na stránce se servery Googlu, aniž by se musel klient stále dotazovat serveru. Toto se hodí, pokud chceme uživatele informovat o nastalé akci, toto se hodí například na hry pro více hráčů a internetové chaty.

Posledním rozšířením je Multitenancy API, to nám dává možnost používat jmenné prostory pro naše data, můžeme je aplikovat na: Datastore, Memcache, Task Queue a Blobstore. Můžeme tak provozovat více oddělených stránek z jedné aplikace a pro každou stránku budeme mít speciální jmenný prostor. Data se tak nebudou překrývat a budou správně oddělena.

\section{Omezení cloudu}
Pokud se rozhodneme naši službu provozovat na cloudu, tak musíme již od návrhu počítat s jistými omezeními a odlišnou strukturou aplikace, než na jakou jsme zvyklí z klasických aplikací. Většina z těchto omezení plyne z požadavku na škálovatelnost aplikací.

Pro většinu programátorů je největším problémem databáze, používá se totiž poměrně nový typ - NoSQL databáze (pro češtinu nejlépe asi sedí překlad: nerelační databáze). Databáze používaná na App Engine se nazívá Big Table\footnote{http://en.wikipedia.org/wiki/BigTable}, jedná se o vícerozměrnou distribuovanou mapu optimalizovanou pro rychlé čtení a pomalejší zápis, protože u běžných aplikací je čtení dat mnohem častější operace. Google navíc toto uložiště využívá i pro své ostatní služby. Naprostá většina dnešních aplikací využívá relační databázi, pravděpodobně jednu z trojice nejpoužívanějších: Oracle, PostgreSQL, MySQL anebo MS-SQL, všechny tyto databáze mají tabulky a pomocí konstrukce JOIN je můžeme navzájem libovolně spojovat. Nevýhoda tohoto řešení ale spočívá v tom, že pro tuto operaci potřebujeme všechny tabulky, které v dotazu spojujeme. V praxi se tedy používá samostatný stroj jen pro databázi. U škálovatelných aplikací, se ale nemůžeme spolehnout na to, že jsou všechny tabulky na jednom místě, mohou totiž být v různých datacentrech na různých kontinentech. Řešením je tedy ukládání všech potřebných dat do jedné tabulky anebo přiřazování do skupin, které se budou spojovat a databáze se sama postará o to, aby byla data uložena ve stejném datacentru. %[sloupce] 
App Engine proto používá speciální dotazovací jazyk šitý na míru tomuto uložišti: GQL - Google Query Language\footnote{http://code.google.com/appengine/docs/python/datastore/gqlreference.html},
což je podmnožina SQL jazyka pro App Engine Datastore. Nenajdeme v něm samozřejmě operátor JOIN a s ním spojené konstrukce.

Mezi další omezení patří žádná anebo omezená možnost vyhledávání nad sloupci databáze. Není totiž zaručeno jakou strukturu sloupců bude tabulka mít. Toto jde obejít vytvořením speciální tabulky se slovy a v kterých sloupcích se vyskytují, ale je to dosti složité a musíme se o vše starat sami. Pokud potřebujeme vyhledávat na internetové stránce, je jednodušší použít internetový vyhledávač například Google, Bing anebo český Seznam. Všechny jmenované mají nástroj pro vyhledávání podle domény, takže stačí jen přidat formulář na naše stránky. Pokud potřebujeme vyhledávát v našich interních datech budeme muset použít nějaké rozsáhlejší řešení.

\section{Vývoj pro App Engine}
Pokud se rozhodneme vytvářet naše aplikace pro App Engine, tak máme k dispozici poměrně vyspělou infrastrukturu. Google oficiálně podporuje Eclipse plugin pro App Engine, %[odkaz] 
ale dostupné jsou i plně funkční pluginy pro NetBeans IDE %[odkaz] 
a také pro vývojové prostředí IntelliJ IDEA. Pomocí těchto pluginů můžeme jednoduše vyvíjet aplikaci na našem domácím stroji bez nutnosti připojení k internetu. Součástí je totiž jednoduchý webový server simulující App Engine, jedná se o upravený Jetty server.%[odkaz]
Máme zde úplně stejné API jako na produkčním serveru a pomocí URL \verb|http://localhost/_ah/admin| máme k dispozici jednoduchou administraci obsahující správce Datastore, správce Task Queues a další. Data se lokálně ukládají do souboru \verb|.bin| přímo ve složce \verb|/build| projektu. Deploy na lokální prostředí je stejný jako u jakéhokoliv jiného serveru, tedy pomocí tlačítka v IDE, anebo můžeme použít některý z buildovacích nástrojů, jako například ANT anebo Maven. Pro upload přímo na produkční prostředí je možné také použít přímo plugin, stejně tak jako je integrováno tlačítko pro lokání upload, tak je zde možnost uploadu přímo na App Engine. Vše probíhá nahráním výsledného \verb|war| souboru aplikace na speciální URL. Pokud bychom chtěli integrovat nahrávání do jiného nástroje, máme možnost provést upload pomocí skriptu pro příkazovou řádku. Ten provádí upload pomocí \verb|jar| souboru, takže není problém celý deployment integrovat do naší infrastruktury. Dále můžeme mít libovolný počet verzí aplikace, všechny jsou na URL \verb|verze.jmeno-aplikace.appspot.com|, kde verze je jakýkoliv řetězec definovaný ve \verb|web.xml| a výchozí možnost se nastavuje v administraci přímo na App Enginu. Máme zde navíc oproti localhostu\footnote{označení serveru, který běží na našem lokálním počítači} mnoho různých nastavení a statistik. Pro každou aplikaci, kterých můžeme k jednomu Google účtu mít až deset je zde podrobný přehled návštěv a spotřebovaných prostředků, počet aktivních instancí, logy, přehled a správa Datastore, nastavení aplikace a další.

\section{Zajímavé aplikace}
Nyní představím zajímavé aplikace a stránky, které můžeme na App Engine cloudu najít. Nacházejí se zde zajímavé experimenty, jako například běh různých jazyků nad JVM (Java Virtual Machine) až po stránky s velkým zatížením. Nejzajímavější z nich je pravděpodobně stránka \verb|www.officialroyalwedding2011.org|, založená k příležitosti svatby anglického prince Williama a Catherine Middleton. V pátek 29. dubna, tedy v den oddání, bylo na hosting generováno 2~000 požadavků za vteřinu a dohromady bylo zobrazeno 15 milionů stránek od 5,6 milionu uživatelů. I přes tento nápor stránka běžela bez problémů a bez ovlivnění více jak 200~000 dalších aplikací běžících na stejném cloudu, které všechny dohromady za den vygenerovaly více jak 1,5 miliardy stránek. \cite{royal-wedding}

Další podobnou zkouškou pro App Engine byla aplikace Google Moderator. Tato aplikace běžela dva dny v březnu roku 2009 na stránce \verb|www.whitehouse.gov|. Jednlo se o hlasovací systém určený pro obyvatele USA, kde může kdokoliv vložit svůj dotaz a další uživatelé pak hlasují o tom, které dotazy jsou nejlepší. Vítězné otázky byly dne 29. března zodpovězeny prezidentem Barackem Obamou. Během 48 hodin zadalo 92~934 uživatelů 104~073 otázek a ohodnotilo je 3~605~984 hasy. Při nejvyšší zátěži obsloužil App Engine 700 dotazů za vteřinu.
%[obrazek-prezentace]
%[http://googlecode.blogspot.com/2009/04/google-developer-products-help.html]
%[http://techcrunch.com/2009/04/07/happy-birthday-app-engine-its-been-a-good-year/]
%[http://techcrunch.com/2009/03/24/white-house-using-google-moderator-for-town-hall-meeting/]

Nyní bych rád zmínil některé zajímavá a užitečné aplikace, které jsou ideálí pro umístění v cloudu. Jednou z nich je i Socialwok (\verb|www.socialwok.com|), jedá se o obdobu facebooku pro práci, můžeme zde sdílet naši práci v Google Apps (Docs, Calendar, Spreadsheet) se spolupracovníky a ti mohou do našich dokumentů zasahovat. Jedná se o zajímavou myšlenku a praktické využití sociálních sítí.

Podobnou službou je i Giftag (\verb|www.giftag.com|), ta umí uložit část webové stránky a tu pak sdílet s dalšími. Existuje doplněk pro internetové prohlížeče, který zjednoduší uložení stránky. Všechny uložené části navíc můžeme organizovat a přidávat do seznamů. Tato služba nám může pomoci, pokud pracujeme na výzkumu anebo potřebujeme udělat prezentaci na které pracuje více lidí.
 
Nyní z jiného soudku, největší uplatnění získaly cloudy díky sociálním sítím, další služba je určena právě pro ně. Jedná se o BuddyPoke! (\verb|www.buddypoke.com|), umožňuje nám dát si na náš profil trojrozměrný obrázek postavičky s popisem jak se cítíme. Tuto aplikaci můžeme najít na všech používaných sociálních sítích, které vkládání obrázků dovolují, tedy: Facebook, Orkut, MySpace, hi5, Netlog, Ning a dalších. Tato aplikace nemá žádný přínos, ale díky velkému množství podporovaných sociálních sítí a jejich velké oblibě v poslední době je tato služba velice úspěšná.

Poslední zmíněná aplikace je naopak velmi přínosná. Mnoho vývojářů by rádo vidělo na App Engine podporu pro skriptovací jazyk PHP, ten se bohužel vývojáři v nejbližším čase přidat neplánují. Projekt Quercus (\verb|quercus.caucho.com|) umožňuje právě běh PHP na JVM a vývojáři připravili speciální verzi pro App Engine, kde můžeme spouštět PHP s podporou některých Java frameworků.

Další zajímavé a úspěšné projekty jsou na stránce: \verb|http://code.google.com/appengine/casestudies.html|

%*****************************************************************************
\cleardoublepage
\chapter{Implementace: Vývoj v App Engine}

\section{Výběr cloudu}
Pro praktickou ukázku a prověření implementace jsem vybral cloud App Engine od společnosti Google. Nejdůležitějším důvodem pro mne byla možnost nahrát plnohodnotnou aplikaci bez nutnosti vynaložení jakýchkoliv nákladů na hosting. Nastavené kvóty od kterých je nutné platit jsou vysoké (odpovídají několika stovkám tisíců požadavkům za den) a stále se spostupně zvyšují. Je tedy možné s tímto cloudem libovolně experimentovat a nahrávat různé testovací aplikace. Mnoho Java vývojářů hledá pro svoje malé aplikace hosting, kde by nemuseli platit, anebo mohli nahrát více aplikací. To je ale problém webového Java světa. Server počítá pouze s jedinou aplikací a je často nutné si pamatovat prostředky a další zdroje v rámci aplikace mezi požadavky. Javovský server tak počítá s tím, že si alokuje většinu dostupné paměti systému. Výhodou je, že nemusíme při každém požadavku vytvářet nové spojení do databáze a k ostatním prostředkům. Naproti tomu u skriptovací jazyků (jako například \verb|PHP| nebo \verb|Python|), se při každém požadavku provede celý kód znovu. Na serveru tak může běžet několik aplikací a zároveň se neovlivňují. V praxi se toho běžně využívá a proto je možné provozovat velké množství takovýchto aplikací na jediném hosting. App Engine nám umožňuje stejný princip pro Javu, běh více aplikací na stejném hardware. Cena za tuto možnost je nutnost uvolnění zdrojů z nepoužívaných aplikací, aby nezabírali místo ostatním. To App Engine sám hlídá a po určité době neaktivity aplikaci odstaví a nahraje aktivní.

\section{Aplikace}
Cílem této bakalářské práce je ověřit možnosti a omezení škálovatelných aplikací. Nejdůležitějším hlediskem je celková rychlost a odezva aplikace v závislosti na počtu souběžných požadavků. Tedy jak bude aplikace a celý cloud reagovat na zvýšený nápor požadavků, jak se s tím cloud vyrovná a zda bude služba stále poutžitelná. Také jsem se zaměřil na rychlost čtení a ukládání do Datastore, pokud je klasická aplikace vystavena vysoké zátěži, je právě databáze úzkým hrdlem (takzvaný \emph{bottleneck}) a přestává stíhat jako první. 

Kromě ověření rychlosti a škálovatelnosi bylo také cílem vytvořit rozsáhlejší aplikaci a vyzkoušet všechny úskalí, které nám platforma App Engine staví do cesty. To znamená implementace běžných požadavků na aplikace, například M:N\footnote{Many-to-Many - druh relace, kde více entit může být spojeno s více entitami (M:N), v relačních databázích je pro propojení použita spojovací tabulka} relace mezi entitami, zamykání dat proti přepsání a další požadavky, které jsou na aplikace běžně kladeny. Po vytvoření tohoto základu a překonání všech zádrhelů již bude jednoduché takovouto aplikaci upravit pro jiné požadavky, anebo rozšířit o nové funkce.

Jako nejvhodnější řešení pro vyzkoušení implementace jsem vybral jednoduchý systém pro správu obsahu (CMS - Content Management System). Můžeme si tak vytvořit webovou  stránku s plnou administrací, tedy s možností přidávání, úpravy a mazání stránek. Ke každé stránce můžeme přidat šablonu. Mimo samotných stránek lze přidávat, upravovat a mazat i novinky, jedná se o kratší zprávy, pro které je zbytečné vytvářet samostatnou stránku. Ke každé stránce lze také přiřadit štítek (tag), tato možnost nahrazuje kategorie. Výhodou oproti kategoriím, kde může být jeden článek pouze v jedné kategorii je, že článek může být označen více štítky a zároveň jeden štítek může být přiřazen k více článkům. Z těchto údajů pak můžeme generovat takzvané \emph{tag cloudy} (oblaky štítků)\footnote{http://en.wikipedia.org/wiki/Tag\_cloud}, kde velikost štítku určuje jeho význam, čim více článků štítek označuje, tím je důležitější a výraznější. Z hlediska implementace se jedná o vazbu M:N, která se v NoSQL databázích musí implementovat složitěji než v relačních a bylo potřeba vymyslet nejlepší řešení.

\section{Výběr frameworku: Slim3}
Než jsem začal aplikaci psát, zjišťoval jsem, které knihovny a frameworky\footnote{Framework je ucelený soubor knihoven a kódu pomáhající vytvořit aplikaci} bych mohl použít na ulehční práce. Kvůli omezením nejsou podporovány všechny knihovny a některé potřebují speciální úpravy anebo nastavení. Další nevýhodou je, že čím je náš kód rozsáhleší, tím déle bude trvat načítání, pokaždé když bude App Engine nahrávat aplikaci do aktivního stavu. Tímto jsem eliminoval známé, ale velké frameworky jako jsou Spring\footnote{Spring Framework - http://www.springsource.org/} a Seam\footnote{Seam Framework - http://seamframework.org/}. Poté jsem hledal mezi menšími, ale žádný nevyhovoval úplně potřebám App Engine. Bohužel je tento cloud relativně mladý a tak pro něj neexistují frameworky, anebo jsou málo známé. Myslel jsem tedy že použiji přímočaré řešení pomocí servletů a naprogramuji aplikaci od základů. Naštěstí jsem ale náhodou narazil na framework Slim3\footnote{Slim3 - http://sites.google.com/site/slim3appengine/} určený přesně pro potřeby App Engine.

Slim3 je MVC\footnote{Model View Controller - http://en.wikipedia.org/wiki/Model-view-controller}
framework optimalizovaný pro App Engine, přináší rozšíření i jednodušší práci s Datastore API a další vylepšení. Jeho hlavní koncepty jsou: \emph{simple} (jednoduchý) a \emph{less is more} (méně je více), tedy že jednoduchost a srozumitelnost vedou ke správnému designu. Framework se drží Paretova pravidla, tedy že 80\% aplikace, pramení z 20\% práce, takže se snaží co nejvíce zjednodušit oněch 20\% a ve zbytku nechává programátorovi volnost.

Nejvíce nám framework pomáhá při práci s Datastore. Pokud s uložištěm chceme pracovat, máme na App Engine možnost použít klasické JPA\footnote{JPA - Java Presistence API - http://en.wikipedia.org/wiki/Java\_Persistence\_API} s těmito omezeními: nemůžeme použít vztah \verb|many-to-many|, v dotazu nemůžeme použít \verb|JOIN| a agregační funkce ( \verb|GROUP BY|,  \verb|HAVING|,  \verb|SUM|,  \verb|AVG|,  \verb|MAX| a  \verb|MIN|) a polymorfické dotazy (slouží k získání podtřídy). Druhou možností je použít JDO\footnote{JDO - Java Data Obects - http://en.wikipedia.org/wiki/Java\_Data\_Objects}, jedná se o obecněší obdobu JPA, kde můžeme náš doménový model ukládat do různých uložišť - relačních, nerelačních, objektových databází,  XML i do obyčejných souborů. Programátorská práce je stejná jako s JPA, definujeme model pomocí anotací\footnote{Anotace v Jazyce java představují speciální konstrukce začínající @ a přidávající ke kódu dále zpracovatelné metainformace} a dotazujeme se pomocí objektu \verb|Query|. Poslední možností je použít přímo Datastore API, které je optimalizované pro práci s BigTable, Dovoluje nám měnit strukturu entit za běhu. Znamená to, že se nemůžeme spolehnout na to, že je daný sloupec v entitě obsažen, ani na jeho typ. Sloupce u entity je vlastně kolekce objektů, v Javě by BigTable vypadala následovně: \verb|Map<Key, Set<Object>>|. Toto je důležité uvědomit si a může nám to přinést potíže v začátcích, dobrou radou je ukládat si přímo do objektu i verzi schématu, při změně v aplikaci se totiž schémata uložených entit nezmění. 

Další změnou jsou transakce, pro správné fungování potřebujeme, aby byl všechny typy entity používané v transakci na jednom stroji. Toto může být problém, jelikož Google má několik datacenter a není tak jisté, že budou všechny entity pohromadě. Kvůli tomuto zavádí App Engine takzvané \emph{entity groups} (skupiy entit). Entity přiřadíme do stejné skupiny tak, že nové entitě nastavíme tzv. \emph{parent entity} (rodičovskou entitu), její klíč pak bude složen z kílče samotné entity a klíče rodičovské entity. Pokud se pokusíe provést operace v tansakci na entitách z rozdílných skupin, vyhodí App Engine výjimku.


%*****************************************************************************
\cleardoublepage
\chapter{Testování: Porovnání rychlosti práce s Datastore}

\section{Hlediska a způsob testování}
Pro otestování cloudu jsem vybral dvě důležitá hlediska, prvním byla rychlost Datastore API a provádění operací: čtení, vkládání, úprava a mazání záznamů v závislosti na použitém přístupu. Testovány byly tyto možnosti: JPA, JDO a Datastore API. Druhým typem testů bylo zatížení celé aplikace a reakce cloudu. Vytvořil jsem velké množství stránek a poté jsem pomocí nástroje Apache Bench \footnote{Apache Bench je nástroj pro příkazovou řádku určený k měření výkonu serverů -http://httpd.apache.org/docs/2.2/programs/ab.html}  posílal na App Engine požadavky. Toto testování by mělo reflektovat situaci, kdy se o dané stránce dozví v krátkém čase velký počet návštěvníků, například vyjde článek nebo reportáž, spuštění velké kampaně, anebo například použití při propagaci jednorázové události jako jsou volby.

Všechny testy probíhaly přímo na App Engine. Sice existuje server pro lokální vývoj, ale není zcela stejný jako cloud, jedná se jen o webový server Jetty obohacený o API služeb cloudu. Důležitým důvodem také je, že aplikace běží spolu s dalšími stránkami a navzájem se tedy ovlivňují. Cílem test bylo simulovat co nejvíce reálné prostředí, abychom mohli odhadnout chování aplikace.  

Pokud je aplikace nějáký čas neaktivní, tak se odnahraje aby systém šetřil prostředky pro jiné aplikace. Toto nahrávání zabírá nezanedbatelný čas a mohlo by ovlivnit výsledky testů. Před každým testem byla tedy aplikace navštívena, aby bylo zajištěno, že bude připravena a nebude potřeba ji znovu nahrávat.

\section{Testování Datastore}
Pro testování samotného uložiště jsem napsal jednoduchou aplikaci s názvem \emph{TodoList}. Tato aplikace slouží k jednoduché evidenci úkolů, ty můžeme přidávat, upravovat a mazat. Model aplikace obsahuje jednu entitu \verb|TodoEntity| a Datastore bylo naplněno několika tisíci záznamy. Pro každou z testovacích možností, tedy JPA, JDO a Datastore API, jsem vytvořil entitu s odpovídajícími anotacemi a DAO třídu \verb|TodoDAO| pro práci s Datastore. Zbytek aplikace jsem ponechal vždy stejný a při testech jsem měnil jen tuto vrstvu. Při testování byl měřen jen čistý čas operace, aby nebyly výsledky zkresleny síťovým zpožděním a dalšími faktory. Zdrojové kódy všech aplikací i s ukázkami jsou volně k dispozici\footnote{http://code.google.com/p/ae-cms/wiki/Aplikace}

Datastore API je optimalizováno přesně pro uložiště BigTable, používané na App Engine. Jedná se o nejpřímočařejší způsob a tak bylo očekáváno, že i tento způsob práce vyjde z testů nelépe. JPA a JDO používají anotace pro zjednodušení celé práce, ale vše je vykoupeno nutností tato data zpracovávat a převádět. To samozdřejmě celou práci zpomaluje a dá se očekávat, že tyto přístupy vyjdou z testu rychlosti práce podobně, ale pomaleji než Datastore API.

První kolo testování probíhalo v odpoledních hodinách a kvůli vlivu ostatních aplikací byly některé výsledky výrazně odlišné, čož celé porovnání zkreslovalo. Proto byly testy provedeny znovu kolem půlnoci našeho času, kde k takto výrazným odchylkám nedocházelo, ale přesto byl vliv ostatních aplikací někdy znatelný.

\section{Test výběru}
Prvním testem bylo vybrání 1000 entit z Datastore pomocí všech způsobů. Pro vybírání záznamů z uložiště je jasným favortiem Datastore API, toto rozhraní je pro výběr optimalizován, jelikož výběr bývá u většiny aplikací několikanásobně častější než ostatní operace. Na druhém místě se umístil přístup pomocí JPA a o něco pomalejší je JDO.

\begin{table}[h]
\centering
\caption{Tabulka porovnání výběru 1~000 entit }\label{tab:select}
\begin{tabular}{|l|c|c|c|c|c|c|c|c|c|c|c|}
   \hline
	& 1		& 2		& 3		& 4		& 5		& 6		& 7		& 8		& 9		& 10		& průměr \\
   \hline
Datastore API	& 1591	& 1649	& 1736	& 1708	& 2292	& 1752	& 1768	& 1645	& 1738	& 1645	& 1752.4 \\
JPA	& 2677	& 2438	& 2518	& 2345	& 2321	& 2337	& 2253	& 2304	& 2203	& 2346	& 2374.2 \\
JDO	& 2447	& 2345	& 2341	& 2619	& 2799	& 2912	& 2581	& 2632	& 2411	& 2192	& 2527.9 \\
   \hline
\end{tabular}
\end{table}

\begin{figure}[h]
\begin{center}
\includegraphics[width=6.5in]{figures/select.png}
\caption{Graf porovnání výběru 1~000 entit}
\label{fig:select}
\end{center}
\end{figure}

\section{Test vkládání}
Pro vkládaní záznamů již nejsou výsledky tolik rozdílné. Vítězem je znovu Datastore API, ale hned v těsném závěsu se umístil přístup pomocí JDO a hned za ním JPA. 

\begin{table}[h]
\centering
\caption{Tabulka porovnání vkládání 100 entit }\label{tab:insert}
\begin{tabular}{|l|c|c|c|c|c|c|c|c|c|c|c|}
   \hline
	& 1		& 2		& 3		& 4		& 5		& 6		& 7		& 8		& 9		& 10		& průměr \\
   \hline
Datastore API	& 2421	& 1931	& 1810	& 1663	& 1637	& 1445	& 1543	& 1488	& 1585	& 1578	& 1710.1 \\
JPA	& 2429	& 2178	& 2120	& 2044	& 2255	& 2023	& 1961	& 2698	& 1973	& 1878	& 2155.9 \\
JDO	& 1891	& 1881	& 1851	& 1696	& 1764	& 2012	& 1781	& 1834	& 1765	& 1826	& 1830.1 \\
   \hline
\end{tabular}
\end{table}

\begin{figure}[h]
\begin{center}
\includegraphics[width=6.5in]{figures/insert.png}
\caption{Graf porovnání vkládání 100 entit}
\label{fig:insert}
\end{center}
\end{figure}

\section{Test úprava}

Při úpravě entity překvapivě JPA i JDO pracovaly rychleji než Datastore API. Nejrychlejší byl nyní JDO přístup. Pomalost Datastore API byla pravděpodobně způsobena implementcí DAO vrstvy. Nejdříve byla entita vybrána z databáze, poté byly změněny hodnoty a následně celá entita znovu uložena.

\begin{table}[h]
\centering
\caption{Tabulka porovnání úpravy 100 entit }\label{tab:update}
\begin{tabular}{|l|c|c|c|c|c|c|c|c|c|c|c|}
   \hline
	& 1		& 2		& 3		& 4		& 5		& 6		& 7		& 8		& 9		& 10		& průměr \\
   \hline
Datastore API	& 2309	& 5809	& 2442	& 2245	& 2283	& 2199	& 3407	& 2908	& 2085	& 2217	& 2501.2 \\
JPA	& 2139	& 2171	& 2219	& 2135	& 2026	& 2234	& 1997	& 6232	& 2191	& 2181	& 2162 \\
JDO	& 1704	& 1775	& 1950	& 1703	& 1735	& 1870	& 1752	& 1711	& 1889	& 1719	& 1769.4 \\
   \hline
\end{tabular}
\end{table}

\begin{figure}[h]
\begin{center}
\includegraphics[width=6.5in]{figures/update.png}
\caption{Graf porovnání úpravy 100 entit}
\label{fig:update}
\end{center}
\end{figure}


\section{Test mazání}

Pro mazání bylo Datastore API výrazně rychlejší. JPA a JDO na tom byly s rychlostí podobně, ale JPA vyšlo o něco lépe.

\begin{table}[h]
\centering
\caption{Tabulka porovnání mazání 100 entit }\label{tab:delete}
\begin{tabular}{|l|c|c|c|c|c|c|c|c|c|c|c|}
   \hline
	& 1		& 2		& 3		& 4		& 5		& 6		& 7		& 8		& 9		& 10		& průměr \\
   \hline
Datastore API	& 1591	& 1649	& 1736	& 1708	& 2292	& 1752	& 1768	& 1645	& 1738	& 1645	& 1752.4 \\
JPA	& 2677	& 2438	& 2518	& 2345	& 2321	& 2337	& 2253	& 2304	& 2203	& 2346	& 2374.2 \\
JDO	& 2447	& 2345	& 2341	& 2619	& 2799	& 2912	& 2581	& 2632	& 2411	& 2192	& 2527.9 \\
   \hline
\end{tabular}
\end{table}

\begin{figure}[h]
\begin{center}
\includegraphics[width=6.5in]{figures/delete.png}
\caption{Graf porovnání mazání 100 entit}
\label{fig:delete}
\end{center}
\end{figure}

\section{Porovnání výsledků testů práce s Datastore}
Celkově tedy v testech dopadlo dle očekávání nejrychleji Datastore API. Jedná se o přímočarý přístup k uložišti a nestará se o další transforamce. Bohužel tato rychlost je vykoupena složitější implementací a odlišným způsobem práce. Při úpravě dat bylo podle výsledků testů Datastore API nejpomalejší, to bylo způsobeno odlišným způsobem práce, kdy jsme objekt vybírali z databáze a poté upravovali. Je to dáno rozdílností práce s Datastore API oproti JPA a JDO, pravděpodobně by se nám pomocí optimalizací podařilo dostat výsledky na stejnou úroveň jako ostatní dvě řešení. Každopádně i přesto je Datastore API jasným vítězem. JPA a JDO jsou na tom podl prvedených testů výkonostně podobně. JPA vyniká v rychlosti při čtení a mazání, kdežto JDO je rychlejší při vkládání a úpravě. Práce s těmito dvěma přístupy je podobná a záleží tedy hlavně na zvyku a zkušesnosti programátora, kterou si vybere. Při velkém počtu čtení z Datastore oproti ostatním operacím je z těchto dvou přístupů je z hlediska rychlosti vhodnější využít právě JPA.

%*****************************************************************************
\cleardoublepage
\chapter{Závěr: Zhodnocení}

\section{Jednoduché škálování}
App Engine nám umožňuje psát jednoduše škálovatelné aplikace a poskytuje nám k tomuto účelu zajímavou platformu pro jednoduchý vývoj a nasazení aplikací. Možnost jednoduché škálovatelnosti ale přináší do vývoje některá omezení a rozdílné vývojářské postupy. App Engine motivuje pro využití svých služeb velmi zajímavým business modelem, kde malé a málo využívané aplikace nemusí platit nic a platba za hosting je nutná až po překročení vysokých kvót. Toto je velmi zajímavá možnost pro začínající aplikace, takzvané \emph{start-upy}, kde můžeme spustit projekt ihned a výdaje spojené s hostingem přijdou, až aplikace začne prosperovat.

\section{Zhodnocení porovnání Datastore API, JPA, JDO}
Jako výsledek této bakalářské práce vzniklo několik aplikací. Většina z nich byly jen testovací prototypy na ověření některé z funkčností, anebo pro vyzkoušení práce se službami, které App Engine nabízí. Nejzajímavějšími z nich byly tři aplikace TodoList, pro tři různé využití možnosti práce s uložištěm: Datastore API, JPA a JDO. Tyto aplikace sloužily k porovnání rychlosti každého z těchto přístupů. Nejrychlejším z těchto přístupů byl dle očekávání Datastore API, který je připraven právě pro práci s Datastore, nevýhodou je složitější práce než s JPA a JDO. Dalším důležitým poznatkem je, že uložiště na App Engine je optimalizované pro čtení, takže je výběr dat několikanásobně rychlejší než vkládání, úprava a mazání dat.

Do budoucna by se dal otestovat výběr více závislých entit z uložiště s různými typy relací (\emph{one-to-one}, \emph{one-to-many} a \emph{many-to-many}). Každý ze způsobů práce s uložištěm může využívat jiný typ propojení - muselo by se zařídít, aby byly testy spravedlivé. Všechny typy spojení by musely být v Datastore uloženy stejně.

\section{Zhodnocení porovnání testů zátěže}
Největší a nejpřínostnější aplikací je Content Management System využívající framework Slim3, který je optimalizovaný přesně pro použití na App Engine. Tato aplikace je nejrozsáhlejší a bez problému může být nasazena pro jednoduché stránky. Hlavním přínosem této apikace bylo vyzkoušet reálnou aplikaci s vazbami mezi entitami. Pro tuto aplikaci bylo velmi výhodné použití frameworku, který urychlil vývoj a pomohl usnadnit některé části programu, například optimistické zamykání. Navíc nám tento framework dává dost volnosti, takže z něj můžeme například použít jen část pro práci s uložištěm. 

Tato aplikace byla použita na testování vysoké zátěže na rozsáhlejší a plnohodnotnou aplikaci. Cílem testu bylo zjistit, jak se bude App Engine chovat a jak kvalitní bude možnost rozložení zátěže. Na aplikaci bylo posláno velké množství požadavků a byl měřen celkový čas zpracování požadavků a rychlost odpovědi aplikace.
App Engine si sám podle zatížení aplikace rozhodl, kolik instancí aplikace má být nahráno, aby byly všechny požadavky zpracovány a zátěž byla rovnoměrně rozložena. O toto všechno se stará App Engine sám, navíc rozhodne na které z datacenter aplikaci nahraje. Kvóty touto zátěží byly ovlivněny jen minimálně. Jedině u procesorového času byla vidět větší spotřeba, ale stále zde byly rezervy. 

\section{Osobní přínos}
Největším přínosem pro mne osobně, bylo napsání větší aplikace pro App Engine. Dříve jsem zkoušel jen některé jednoduché aplikace pro vlastní potřebu. Většina z nich sloužila jen pro demonstraci určitého API. Nyní mám praktické zkušenosti s větší aplikací a myslím, že jsem dokázal proniknout do fungování App Engine a dobře znám jeho výhody a nevýhody. Troufnul bych si nyní i na rozsáhlejší aplikaci s reálnou zátěží a plánuji do budoucna takovouto aplikaci napsat. Zajímavá se mi jeví možnost naprogramovat aplikaci pro sociální sítě, kde se zajímavé projekty rozšíří velmi rychle a App Engine je tedy vhodným kandidátem, v opačném případě pokud aplikace nebude pro uživatele zajímavá, budou náklady na hosting nulové.

Kromě práce s App Engine jsem se při vytváření této bakalářské práce naučil efektivně využívat verzovací systém Git\footnote{Git --- http://git-scm.com/} a všechny zdrojové kódy jsou tak k dispozici na verzovacím hostingu Github na adrese: \verb|http://github.com/jskvara|. Dále jsem pro publikování výsledků práce využíval hosting projeků na Google code\footnote{Google code --- http://code.google.com/}, kde je možnost publikovat své výsledky, zdrojové kódy, aplikace ke stažení, dokumentaci projektu ve formátu wiki\footnote{Wiki formát je určen pro publikaci převážně HTML (HyperText Markup Language) obsahu generovaného běžými uživateli, kde je syntaxe pro uživatele přívětivější} a další. Při samotném vytváření textu bakalářské práce, jsem se seznámil se nástrojem \LaTeX, který slouží k profesionální sazbě a tisku dokumentů, což pro mne bylo velmi zajímavé a určitě tento formát v budoucnu použiji znovu.

%*****************************************************************************
% Seznam literatury je v samostatnem souboru reference.bib. Ten
% upravte dle vlastnich potreb, potom zpracujte (a do textu
% zapracujte) pomoci prikazu bibtex a nasledne pdflatex (nebo
% latex). Druhy z nich alespon 2x, aby se poresily odkazy.

% originally following specification for bibliography formating was used
%\bibliographystyle{abbrv}

% Here is an improvment by Petr Dlouhy (April 2010).
% It is mainly for supervisors who expect Czech fomrating rules for references
% Additional feature is live url addresses to sources from your pdf file
% It requires the file csplainnat.bst (included in this sample zipfile).

\bibliographystyle{csplainnat}

%bibliographystyle{plain}
%\bibliographystyle{psc}
{
%JZ: 11.12.2008 Kdo chce mit v techto ukazkovych odkazech take odkaz na CSTeX:
\def\CS{$\cal C\kern-0.1667em\lower.5ex\hbox{$\cal S$}\kern-0.075em $}
\bibliography{reference}
}

% M. Dušek radi:
%\bibliographystyle{alpha}
% kdy citace ma tvar [AutorRok] (napriklad [Cook97]). Sice to asi neni  podle ceske normy (BTW BibTeX stejne neodpovida ceske norme), ale je to nejprehlednejsi.
% 3.5.2009 JZ polemizuje: BibTeX neobvinujte, napiste a poskytnete nam styl (.bst) splnujici citacni normu CSN/ISO.

%*****************************************************************************
%*****************************************************************************
\appendix

\chapter{Seznam použitých zkratek}

\begin{description}
\item[API] Application Programming Interface - rozhraní pro programování aplikací %http://cs.wikipedia.org/wiki/API
\item[CEO] Chief executive officer - ředitel
\item[URL] Uniform Resource Locator - jednoznačný identifikátor u webových stránek, například: \verb|http://www.example.com| %\verb|http://en.wikipedia.org/wiki/Uniform_Resource_Locator|
\end{description}

%*****************************************************************************
\chapter{Obsah přiloženého CD}

\begin{figure}[h]
\begin{center}
\includegraphics[width=14cm]{figures/seznamcd}
\caption{Seznam přiloženého CD}
\label{fig:seznamcd}
\end{center}
\end{figure}

Na GNU/Linuxu si strukturu přiloženého CD můžete snadno vyrobit příkazem:\\ 
\verb|$ tree . >tree.txt|\\
Ve vzniklém souboru pak stačí pouze doplnit komentáře.

Z \textbf{README.TXT} (případne index.html apod.)  musí být rovněž zřejmé, jak programy instalovat, spouštět a jaké požadavky mají tyto programy na hardware.

Adresář \textbf{text}  musí obsahovat soubor s vlastním textem práce v PDF nebo PS formátu, který bude později použit pro prezentaci diplomové práce na WWW.

\end{document}

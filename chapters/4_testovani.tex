\chapter{Testování: Porovnání rychlosti práce s Datastore}

\section{Hlediska a způsob testování}
Pro otestování cloudu jsem vybral dvě důležitá hlediska, prvním byla rychlost Datastore API a provádění operací: čtení, vkládání, úprava a mazání záznamů v závislosti na použitém přístupu. Testovány byly tyto možnosti: JPA, JDO a Datastore API. Druhým typem testů bylo zatížení celé aplikace a reakce cloudu. Vytvořil jsem velké množství stránek a poté jsem pomocí nástroje Apache Bench \footnote{Apache Bench je nástroj pro příkazovou řádku určený k měření výkonu serverů -http://httpd.apache.org/docs/2.2/programs/ab.html}  posílal na App Engine požadavky. Toto testování by mělo reflektovat situaci, kdy se o dané stránce dozví v krátkém čase velký počet návštěvníků, například vyjde článek nebo reportáž, spuštění velké kampaně, anebo například použití při propagaci jednorázové události jako jsou volby.

Všechny testy probíhaly přímo na App Engine. Sice existuje server pro lokální vývoj, ale není zcela stejný jako cloud, jedná se jen o webový server Jetty obohacený o API služeb cloudu. Důležitým důvodem také je, že aplikace běží spolu s dalšími stránkami a navzájem se tedy ovlivňují. Cílem test bylo simulovat co nejvíce reálné prostředí, abychom mohli odhadnout chování aplikace.  

Pokud je aplikace nějáký čas neaktivní, tak se odnahraje aby systém šetřil prostředky pro jiné aplikace. Toto nahrávání zabírá nezanedbatelný čas a mohlo by ovlivnit výsledky testů. Před každým testem byla tedy aplikace navštívena, aby bylo zajištěno, že bude připravena a nebude potřeba ji znovu nahrávat.

\section{Testování Datastore}
Pro testování samotného uložiště jsem napsal jednoduchou aplikaci s názvem \emph{TodoList}. Tato aplikace slouží k jednoduché evidenci úkolů, ty můžeme přidávat, upravovat a mazat. Model aplikace obsahuje jednu entitu \verb|TodoEntity| a Datastore bylo naplněno několika tisíci záznamy. Pro každou z testovacích možností, tedy JPA, JDO a Datastore API, jsem vytvořil entitu s odpovídajícími anotacemi a DAO třídu \verb|TodoDAO| pro práci s Datastore. Zbytek aplikace jsem ponechal vždy stejný a při testech jsem měnil jen tuto vrstvu. Při testování byl měřen jen čistý čas operace, aby nebyly výsledky zkresleny síťovým zpožděním a dalšími faktory. Zdrojové kódy všech aplikací i s ukázkami jsou volně k dispozici\footnote{http://code.google.com/p/ae-cms/wiki/Aplikace}

Datastore API je optimalizováno přesně pro uložiště BigTable, používané na App Engine. Jedná se o nejpřímočařejší způsob a tak bylo očekáváno, že i tento způsob práce vyjde z testů nelépe. JPA a JDO používají anotace pro zjednodušení celé práce, ale vše je vykoupeno nutností tato data zpracovávat a převádět. To samozdřejmě celou práci zpomaluje a dá se očekávat, že tyto přístupy vyjdou z testu rychlosti práce podobně, ale pomaleji než Datastore API.

První kolo testování probíhalo v odpoledních hodinách a kvůli vlivu ostatních aplikací byly některé výsledky výrazně odlišné, čož celé porovnání zkreslovalo. Proto byly testy provedeny znovu kolem půlnoci našeho času, kde k takto výrazným odchylkám nedocházelo, ale přesto byl vliv ostatních aplikací někdy znatelný.

\section{Test výběru}
Prvním testem bylo vybrání 1000 entit z Datastore pomocí všech způsobů. Pro vybírání záznamů z uložiště je jasným favortiem Datastore API, toto rozhraní je pro výběr optimalizován, jelikož výběr bývá u většiny aplikací několikanásobně častější než ostatní operace. Na druhém místě se umístil přístup pomocí JPA a o něco pomalejší je JDO.

\begin{table}[h]
\centering
\caption{Tabulka porovnání výběru 1~000 entit }\label{tab:select}
\begin{tabular}{|l|c|c|c|c|c|c|c|c|c|c|c|}
   \hline
	& 1		& 2		& 3		& 4		& 5		& 6		& 7		& 8		& 9		& 10		& průměr \\
   \hline
Datastore API	& 1591	& 1649	& 1736	& 1708	& 2292	& 1752	& 1768	& 1645	& 1738	& 1645	& 1752.4 \\
JPA	& 2677	& 2438	& 2518	& 2345	& 2321	& 2337	& 2253	& 2304	& 2203	& 2346	& 2374.2 \\
JDO	& 2447	& 2345	& 2341	& 2619	& 2799	& 2912	& 2581	& 2632	& 2411	& 2192	& 2527.9 \\
   \hline
\end{tabular}
\end{table}

\begin{figure}[h]
\begin{center}
\includegraphics[width=6.5in]{figures/select.png}
\caption{Graf porovnání výběru 1~000 entit}
\label{fig:select}
\end{center}
\end{figure}

\section{Test vkládání}
Pro vkládaní záznamů již nejsou výsledky tolik rozdílné. Vítězem je znovu Datastore API, ale hned v těsném závěsu se umístil přístup pomocí JDO a hned za ním JPA. 

\begin{table}[h]
\centering
\caption{Tabulka porovnání vkládání 100 entit }\label{tab:insert}
\begin{tabular}{|l|c|c|c|c|c|c|c|c|c|c|c|}
   \hline
	& 1		& 2		& 3		& 4		& 5		& 6		& 7		& 8		& 9		& 10		& průměr \\
   \hline
Datastore API	& 2421	& 1931	& 1810	& 1663	& 1637	& 1445	& 1543	& 1488	& 1585	& 1578	& 1710.1 \\
JPA	& 2429	& 2178	& 2120	& 2044	& 2255	& 2023	& 1961	& 2698	& 1973	& 1878	& 2155.9 \\
JDO	& 1891	& 1881	& 1851	& 1696	& 1764	& 2012	& 1781	& 1834	& 1765	& 1826	& 1830.1 \\
   \hline
\end{tabular}
\end{table}

\begin{figure}[h]
\begin{center}
\includegraphics[width=6.5in]{figures/insert.png}
\caption{Graf porovnání vkládání 100 entit}
\label{fig:insert}
\end{center}
\end{figure}

\section{Test úprava}

Při úpravě entity překvapivě JPA i JDO pracovaly rychleji než Datastore API. Nejrychlejší byl nyní JDO přístup. Pomalost Datastore API byla pravděpodobně způsobena implementcí DAO vrstvy. Nejdříve byla entita vybrána z databáze, poté byly změněny hodnoty a následně celá entita znovu uložena.

\begin{table}[h]
\centering
\caption{Tabulka porovnání úpravy 100 entit }\label{tab:update}
\begin{tabular}{|l|c|c|c|c|c|c|c|c|c|c|c|}
   \hline
	& 1		& 2		& 3		& 4		& 5		& 6		& 7		& 8		& 9		& 10		& průměr \\
   \hline
Datastore API	& 2309	& 5809	& 2442	& 2245	& 2283	& 2199	& 3407	& 2908	& 2085	& 2217	& 2501.2 \\
JPA	& 2139	& 2171	& 2219	& 2135	& 2026	& 2234	& 1997	& 6232	& 2191	& 2181	& 2162 \\
JDO	& 1704	& 1775	& 1950	& 1703	& 1735	& 1870	& 1752	& 1711	& 1889	& 1719	& 1769.4 \\
   \hline
\end{tabular}
\end{table}

\begin{figure}[h]
\begin{center}
\includegraphics[width=6.5in]{figures/update.png}
\caption{Graf porovnání úpravy 100 entit}
\label{fig:update}
\end{center}
\end{figure}


\section{Test mazání}

Pro mazání bylo Datastore API výrazně rychlejší. JPA a JDO na tom byly s rychlostí podobně, ale JPA vyšlo o něco lépe.

\begin{table}[h]
\centering
\caption{Tabulka porovnání mazání 100 entit }\label{tab:delete}
\begin{tabular}{|l|c|c|c|c|c|c|c|c|c|c|c|}
   \hline
	& 1		& 2		& 3		& 4		& 5		& 6		& 7		& 8		& 9		& 10		& průměr \\
   \hline
Datastore API	& 1591	& 1649	& 1736	& 1708	& 2292	& 1752	& 1768	& 1645	& 1738	& 1645	& 1752.4 \\
JPA	& 2677	& 2438	& 2518	& 2345	& 2321	& 2337	& 2253	& 2304	& 2203	& 2346	& 2374.2 \\
JDO	& 2447	& 2345	& 2341	& 2619	& 2799	& 2912	& 2581	& 2632	& 2411	& 2192	& 2527.9 \\
   \hline
\end{tabular}
\end{table}

\begin{figure}[h]
\begin{center}
\includegraphics[width=6.5in]{figures/delete.png}
\caption{Graf porovnání mazání 100 entit}
\label{fig:delete}
\end{center}
\end{figure}

\section{Porovnání výsledků testů práce s Datastore}
Celkově tedy v testech dopadlo dle očekávání nejrychleji Datastore API. Jedná se o přímočarý přístup k uložišti a nestará se o další transforamce. Bohužel tato rychlost je vykoupena složitější implementací a odlišným způsobem práce. Při úpravě dat bylo podle výsledků testů Datastore API nejpomalejší, to bylo způsobeno odlišným způsobem práce, kdy jsme objekt vybírali z databáze a poté upravovali. Je to dáno rozdílností práce s Datastore API oproti JPA a JDO, pravděpodobně by se nám pomocí optimalizací podařilo dostat výsledky na stejnou úroveň jako ostatní dvě řešení. Každopádně i přesto je Datastore API jasným vítězem. JPA a JDO jsou na tom podl prvedených testů výkonostně podobně. JPA vyniká v rychlosti při čtení a mazání, kdežto JDO je rychlejší při vkládání a úpravě. Práce s těmito dvěma přístupy je podobná a záleží tedy hlavně na zvyku a zkušesnosti programátora, kterou si vybere. Při velkém počtu čtení z Datastore oproti ostatním operacím je z těchto dvou přístupů je z hlediska rychlosti vhodnější využít právě JPA.
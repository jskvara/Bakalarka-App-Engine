\chapter{Úvod: Výběr tématu}

\section{Co je to cloud}
Pro mou bakalářskou práci jsem si vybral poměrně nové a nezmapované téma a to popis cloudových služeb. Jedná se o nový způsob hostování internetových aplikací a ukládání dat na webu. U klasického způsobu ukládání dat se použije jeden nebo více nezávislých serverů a na něj se nahraje ve většině případů pouze jedna aplikace. Cloud nám přináší nový přístup, místo abychom měli jeden stroj, použijeme více strojů dohromady, které budou spolupracovat. Aplikace žádný rozdíl nepozná a může zde zároveň běžet mnoho programů. 

Důležitým důvodem k využívání cloudů je větší efektivita využití hardware. Pokud je aplikace málo používaná, například v nočních hodinách, jsou zdroje využívány jinými aplikacemi, které mohou obsluhovat uživatele z jiné části světa, kde je třeba odpoledne. Nebo pokud je aplikace vytížena a nestíhá, může systém spustit další instanci té samé aplikace. O rozložení zátěže a správu počtu aplikací se stará cloud samotný.

Další výhodou je možnost dynamicky navyšovat hardwarové parametry infrastruktury přidáváním strojů bez nutnosti přerušit provoz. Cena hardwaru se postupem času snižuje, podle Mooreova zákona\footnote{Mooreův zákon - http://en.wikipedia.org/wiki/Moore's\_law} z roku 1965 se složitost součástek každý rok zdvojnásobí při zachování stejné ceny. Tento zákon platí dodnes a předpokládá se, že bude platit minimálně do roku 2015 až 2020. S možností dynamicky přidávat hardware budeme připraveni rozšiřovat infrastrukturu podle potřeby. O toto se ale starají společnosti poskytující cloudy, nám tedy odpadá nutnost starat se o hardware. 

Nevýhodou cloudu je, že nemáme hardware plně pod kontrolou a musíme se spolehnout na společnost poskytující hosting. Pokud bychom chtěli provozovat vlastní cloud, museli bychom mít obrovskou infrastrukturu a vyvinout vlastní řešení pro efektivní vytížení zdrojů všech aplikací. Takovouto možnost má jen pár společností na světě, které patří k těm největším.

\section{Motivace pro toto téma}
Hlavní motivací k výběru tohoto tématu bylo, že se v poslení době stává cloud čím dál tím více používanější. Jedná se o úplně nový přístup a tak o cloudech zatím nenajdeme mnoho zdrojů. Přišlo mi zajímavé vyzkoušet a prověřit možnosti jednoho z těchtno cloudů. Jak se bude chovat při vysokém počtu požadavků, kde jsou limity takového cloudu a v neposlední řadě kolik hardwarových prostředků bude při zátěži spotřebováno a jaká bude cena.

\section{Průběh testování}
V mé bakalářké práci jsem ověřil a porovnal rychlost různých řešení práce s uložištěm dat na cloudu. Pro tento test jsem připravil jednoduchou aplikaci. Poté jsem napsal větší a složitější aplikaci, kterou jsem následně otestoval vysokým počtem požadavků. Porovnával jsem jak se bude cloud chovat a jak zařídí rozložení zátěže.
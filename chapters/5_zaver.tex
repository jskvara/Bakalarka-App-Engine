\chapter{Závěr: Zhodnocení}

\section{Jednoduché škálování}
App Engine nás nutí psát jednoduše škálovatelné aplikace a poskytuje nám k tomuto účelu zajímavou platformu pro jednoduchý vývoj a nasazení aplikací. Možnost jednodché škálovatelnosti ale přináší do vývoje některá omezení a rozdílné vývojářské postupy. App Engine motivuje pro využití svých služeb velmi zajímavým business modelem, kde malé a málo využívané aplikace nemusí platit nic a platba za hosting je nutná až po překročení vysokých kvót. Toto je velmi zajímavá možnost pro začínající aplikace, takzvané \emph{start-upy}, kde můžeme spustit projek ihned a výdaje spojené s hostingem přijdou, až aplikace začne prosperovat.

\section{Zhodnocení porovnání Datastore API, JPA, JDO}
Jako výsledek této bakalářské práce vzniklo několik aplikací. Většina z nich byly jen testovací prototypy na ověření některé z funkčností, anebo pro vyzkoušení práce se službami, které App Engine nabízí. Nejzajímavějšími z nich byly tř aplikace TodoList, pro tři různé využití možnosti práce s uložištěm: Datastore API, JPA a JDO. Tyto aplikace sloužily k porovnání rychlosti každého z těchto přístupů. Nejrychlejším z těchto přístupů byl dle očekávání Datastore API, který je připraven právě pro práci s Datastore, nevýhodou je složitější práce než s JPA a JDO. Dalším důležitým poznatkem je, že uložiště na App Engine je optimalizované pro čtení, takže je výběr dat několikanásobně rychlejší než manipulace.

Bylo by zajímavé testovat výběr více závislých entit z uložiště s různými typy relací (\emph{one-to-one}, \emph{one-to-many} a \emph{many-to-many}). Každý ze způsobů práce s uložištěm může využívat jiný typ propojení, muselo by se zařídít, aby byly testy spravedlivé. Všechny typy spojení by musely být v Datastore uloženy stejně.

\section{Zhodnocení porovnání testů zátěže}
Největší a nejpřínostnější aplikací je Content Management System využívající framework Slim3, který je optimalizovaný přesně pro použití na App Engine. Tato aplikace je nejrozsáhlejší a bez problému může být nasazena pro jednoduché stránky. Hlavním přínosem této apikace bylo vyzkoušet reálnou aplikaci s vazbami mezi entitami. Pro tuto aplikaci bylo velmi výhodné použití frameworku, který urychlil vývoj a pomohl usnadnit některé části program, například optimistické zamykání. Navíc nám tento framework dává dost volnosti, takže z něj můžeme například použít jen část pro práci s uložištěm. 

Tato aplikace byla použita na testování vysoké zátěže na rozsáhlejší a plnohodnotnou aplikaci. Cílem testu bylo zjistit, jak se bude App Engine chovat a jak kvalitní bude možnost rozložení zátěže. Na aplikaci bylo posláno velké množství požadavků a byl měřen celkový čas zpracování požadavků a rychlost odpovědi aplikace.
App Engine si sám podle zatížení aplikace rozhodl, kolik instancí aplikace má být nahráno, aby byly všechny požadavky zpracovány a zátěž byla rovnoměrně rozložena. O toto všechno se stará App Engine sám, navíc rozhodne na které z datacenter aplikaci nahraje. Kvóty touto zátěží byly ovlivněny jen minimláně. Jedině u procesorového času byla vidět větší spotřeba, ale stále zde byly rezervy. 